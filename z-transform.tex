\documentclass{article}
\usepackage{amsfonts, amsmath, amsthm}

\title{Z-Transform}
\author{Miguel Manese}
\date{Sept. 21, 2009}

\newcommand{\E}{\mathrm{E}}
\newcommand{\Var}{\mathrm{Var}}
\newcommand{\field}[1]{\mathbb{#1}} % requires amsfonts
\begin{document}
\maketitle

\begin{section}{Z-Transform}
\begin{subsection}{Definitions}
\begin{enumerate}
\item Given a sequence $\{x_{n}\}$, its Z-transform $X(z) = Z\{x_{n}\}$ is 
given by 
  \[ X(z) = \sum_{n} x_{n} z^{-n}. \]
\item Region of convergence (ROC) of $X(z)$ is $\{z \in \field{Z} \mid |X(z)| < 
\infty \}$.
  \begin{enumerate}
  \item if there is a $z^{n}$ term $(n > 0)$ then $+\infty \not\in \text{ROC}$.
  \item if there is a $z^{-n}$ term $(n > 0)$ then $+0 \not\in \text{ROC}$.
  \end{enumerate}
\item Example $x_{n} = x(n) = (\frac{1}{2})^{n} u(n) = \{1, \frac{1}{2},
 \frac{1}{4}, \frac{1}{8}, \ldots \}$
  \begin{align}
  X(z) &= 1 + \frac{1}{2}z^{-1} + \frac{1}{4}z^{-2} + \ldots +
              (\frac{1}{2})^{n} z^{-n} + \ldots \notag\\
       &= \sum_{n=0}^{\infty} (\frac{1}{2})^{n} z^{-n} 
        = \sum_{n=0}^{\infty} (\frac{1}{2} z^{-1})^{n}\notag 
  \end{align}
  which converges to $\frac{1}{1 - \frac{1}{2}z^{-1}}$ if 
  $|\frac{1}{2} z^{-1}| < 1$ or equivalently $|z| > \frac{1}{2}$ and so
  \begin{align}
  X(z) &= \frac{1}{1 - \frac{1}{2}z^{-1}} && \text{ROC: } |z| > \frac{1}{2}
   \notag
  \end{align}
\item ROC of infinite duration signals
  \begin{align}
  |X(z)| &= |\sum_{n} x(n)r^{-n} e^{-i\theta n}| \notag\\
         &\leq \sum_{n} |x(n) r^{-n} e^{-i\theta n}| 
           = \sum_{n} |x(n) r^{n}|\notag\\
         &= \sum_{n= -\infty}^{-1} |x(n) r^{-n}| +
            \sum_{n=0}^{\infty} |\frac{x(n)}{r^{n}}| \notag\\
         &= \sum_{n=1}^{\infty} |x(-n) r^{n}| +
            \sum_{n=0}^{\infty} |\frac{x(n)}{r^{n}}|\notag
  \end{align}
  \begin{enumerate}
  \item For 1st term to converge, $\exists r_{1}$ s.t. $|x(-n) r_{1}^{n}|
   \to 0$ as $n \to \infty$ and so all $r < r_{1}$ will also converge
   (i.e. inside the circle of radius $r_{1}$).
  \item For 2nd term to converge, $\left|\frac{x(n)}{r_{2}^{n}}\right| \to 0$
   as $n \to \infty$ and so are for all $r > r_{2}$ (i.e. outside the circle
   of radius $r_{2}$).
  \item Together, ROC is $r_{1} > r > r_{2}$.
  \end{enumerate}
\item $X(z)$ and ROC uniquely identifies $x_{n}$. If $x_{n}$ is finite,
then $X(z)$ alone is enough (ROC everywhere except possibly 0 and $\infty$).
\item Inversion using complex integration over a contour.
\end{enumerate}
\end{subsection}

\begin{subsection}{Properties}
\begin{enumerate}
\item Linearity : $Z\{\alpha x_{1}(n) + \beta x_{2}(n)\} = 
  \alpha X_{1}(z) + \beta X_{2}(z)$, ROC: intersection of ROC of
  $X_{1}(z)$ and $X_{2}(z)$.
\item Time-shifting: $Z\{x(n-k)\} = z^{-k} X(z)$, ROC: same except at $z=0$.
\item Scaling: $Z\{a^{n} x(n)\} = X(\frac{z}{a})$, ROC: 
  $|a|r_{1} < |z| < |a|r_{2}$.
  \begin{proof}
  \begin{align}
  Z\{a^{n}x(n)\} &= \sum_{n} a^{n}z^{-n} = \sum{n} x(n) (a^{-1}z)^{-1}\notag\\
    &= X(\frac{z}{a}) \notag
  \end{align}
  \noindent If ROC of $X(z)$ is $r_{1} < |z| < r_{2}$, then ROC of 
  $X(\frac{z}{a})$ is $r_{1} < |za^{-1}| < r_{2}$ or equivalently 
  $|a|r_{1} < |z| < |a|r_{2}$.
  \end{proof}
\item Time-reversal: $Z\{x(-n)\} = X(z^{-1})$, ROC: $\frac{1}{r_{2}} < |z| <
  \frac{1}{r_{1}}$.
\item Differentiation in Z-domain: $Z\{n x(n)\} = -z 
  \frac{\text{d} X(z)}{\text{d}z}$
  \begin{proof}
  \begin{align}
  -z \frac{\text{d}X(z)}{\text{d}z} &= -z \sum_{n} x(n) (-n)z^{-n-1} 
    = (-z) (-z)^{-1} \sum_{n} (n x(n)) z^{-n} \notag\\
    &= Z\{n x(n)\} \notag
  \end{align}
  
%  -z \sum_{n} 
%        (\sum_{k} x_{1}(k) x_{2}(n-k)) z^{-n} \notag\\
%     &= -z \sum_{k} x_{1}(k) (\sum_{n} x_{2}(n-k) z^{-n}) \notag\\
%     &= -z X_{2}(z)
  \end{proof}
\item Convolution: $Z\{x_{1}(n) * x_{2}(n)\} = X_{1}(z)X_{2}(z)$
  \begin{proof}
  \begin{align}
  Z\{x_{1}(n) * x_{2}(n)\} &= \sum_{n} \left(\sum_{k} x_{1}(k) x_{2}(n-k)\right)
     z^{-n} \notag\\
     &= \sum_{k} x_{1}(k) \left(\sum_{n} x_{2}(n-k) z^{-n}\right)\notag\\
     &= X_{2}(z) \sum_{k} x_{1}(k) z^{-k} && \text{(Time Shift property)}
    \notag\\
     &= X_{2}(z) X_{1}(z) = X_{1}(z) X_{2}(z)\notag
  \end{align}
  \end{proof}
\item Correlation of 2 sequences: if $r_{x_{1},x_{2}}(l) = 
  \sum_{n} x_{1}(n) x_{2}(n-l)$ then $R_{x_{1},x_{2}}(z) = 
  Z\{r_{x_{1},x_{2}}\} = X_{1}(z) X_{2}(z^{-1})$ using time-reversal and
  convolution properties.
\item Multiplication in time domain:
  \[ Z\{x_{1}(n) x_{2}(n) \} = \frac{1}{2 \pi i} \oint X_{1}(v) 
      X_{2}(\frac{z}{v}) v^{-1} dv \]
  \noindent ROC: $r_{1l} r_{2l} < |z| < r_{1u} r_{2u}$, where $r_{1l} < |z| <
  r_{1u}$ is the ROC of $x_{1}(n)$ and similarly for $x_{2}(n)$.
\item Parseval's relation - 
  \[ Z\{\sum_{n} x_{1}(n) x_{2}^{*}(n) \} = \frac{1}{2 \pi i} \oint X_{1}(v) 
      X_{2}^{*}(\frac{1}{v^{*}}) v^{-1} dv \]

\end{enumerate}
\end{subsection}
\end{section}

\begin{section}{Rational Z-Transforms}
\begin{enumerate}
\item Form
  \begin{align}
  X(z) &= \frac{N(z)}{D(z)} = \frac{\sum_{k=0}^{M} b_k z^{-k} }
                                   {\sum_{k=0}^{N} a_k z^{-k} }
        = \frac{b_{0}z^{-M}}{a_{0} z^{-N}}
          \frac{\sum_{k=0}^M \frac{b_{k}}{b_{0}} z^{M-k}}
               {\sum_{k=0}^N \frac{a_{k}}{a_{0}} z^{N-k}} \notag\\
       &= G z^{N-M} \frac{\prod_{k=1}^M (z - z_{k})}
                         {\prod_{k=1}^N (z - p_{k})}\notag
  \end{align}
  \noindent where $z_{k}$ are the \emph{zeros} and $p_k$ are the \emph{poles}.
\item The number of zeros and poles, including those at 0 and $\infty$, are
  always equal. 
\item ROC should not contain poles.
\end{enumerate}
\end{section}

\begin{section}{Pole Locations}
Assume the series $\{x_{n}\} = \{\ldots, x_{-1}, x_{0}, x_{1}, \ldots\}$ 
is the transfer function of the \emph{system} $X$, and its z-transform is 
$X(z) = \sum_{n=-\infty}^{\infty} x_{n} z^{-n}$.
\begin{enumerate}
\item If the series has a finite \emph{support}, then it does not have poles
besides at $z = 0$.
\item The system $X$ is called \emph{causal} if 
$X(z) = \sum_{n=0}^{\infty} x_{n} z^{-n}$ (i.e. zero for negative integers). 
This is because of $X$ is the transfer function, then convolving it with
an input will only involve past values of the input and so is causal.
\item If the series has infinite support, then it can have poles. Common
sense in the finite case does not necessarily extend to the infinite case,
even for linear stuffs. It is is possible that the series is a 
ratio of two polynomials. Then the zeros of the denominator polynomial
are the poles of the series.
\item Implications of the pole location for causal, infinite series
  \begin{enumerate}
  \item Pole at $|z| < 1$ - this is fine, because $|1/z| > 1$ and the sum
  explodes and is unstable
  \item Pole at $|z| = 1$ - this means $\sum_{n=0}^{\infty} |x_{n}| = \infty$,
  and is \emph{semi-stable}
  \item Pole at $|z| > 1$ - this means unstable, because $|1/z| < 1$ and
  the series explodes faster than $|1/z|$ vanishes (i.e., it is large enough
  to overcome the influence of a very small $|1/z^{N}|$ for large $N$).
  \end{enumerate}
\noindent Therefore, for stable systems we want the poles to be $|z| < 1$. 
If the convention used is $X(z) = \sum_{n=0}^{\infty} x_{n} z^{n}$, then
we want the poles at $|z| > 1$.\\
For ARMA $\phi(B)X_{n} = \theta(B)W_{n}$, we want the zeros of $\phi(B)$ to
be \emph{outside of the radius-1 circle} (because we use the second convention)
so that the poles, when \emph{transposed} to the other side, are also
outside the circle.
\end{enumerate}
\end{section}
\end{document}
