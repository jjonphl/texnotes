\documentclass{article}
\usepackage{amsfonts, amsmath}

\title{Z-Transform}
\author{Miguel Manese}
\date{Sept. 21, 2009}

\newcommand{\E}{\mathrm{E}}
\newcommand{\Var}{\mathrm{Var}}
\begin{document}
\maketitle

\begin{section}{Z-Transform}
\end{section}

\begin{section}{Polynomial Fraction}
\end{section}

\begin{section}{Pole Locations}
Assume the series $\{x_{n}\} = \{\ldots, x_{-1}, x_{0}, x_{1}, \ldots\}$ 
is the transfer function of the \emph{system} $X$, and its z-transform is 
$X(z) = \sum_{n=-\infty}^{\infty} x_{n} z^{-n}$.
\begin{enumerate}
\item If the series has a finite \emph{support}, then it does not have poles
besides at $z = 0$.
\item The system $X$ is called \emph{causal} if 
$X(z) = \sum_{n=0}^{\infty} x_{n} z^{-n}$ (i.e. zero for negative integers). 
This is because of $X$ is the transfer function, then convolving it with
an input will only involve past values of the input and so is causal.
\item If the series has infinite support, then it can have poles. Common
sense in the finite case does not necessarily extend to the infinite case,
even for linear stuffs. It is is possible that the series is a 
ratio of two polynomials. Then the zeros of the denominator polynomial
are the poles of the series.
\item Implications of the pole location for causal, infinite series
  \begin{enumerate}
  \item Pole at $|z| < 1$ - this is fine, because $|1/z| > 1$ and the sum
  explodes and is unstable
  \item Pole at $|z| = 1$ - this means $\sum_{n=0}^{\infty} |x_{n}| = \infty$,
  and is \emph{semi-stable}
  \item Pole at $|z| > 1$ - this means unstable, because $|1/z| < 1$ and
  the series explodes faster than $|1/z|$ vanishes (i.e., it is large enough
  to overcome the influence of a very small $|1/z^{N}|$ for large $N$).
  \end{enumerate}
\noindent Therefore, for stable systems we want the poles to be $|z| < 1$. 
If the convention used is $X(z) = \sum_{n=0}^{\infty} x_{n} z^{n}$, then
we want the poles at $|z| > 1$.\\
For ARMA $\phi(B)X_{n} = \theta(B)W_{n}$, we want the zeros of $\phi(B)$ to
be \emph{outside of the radius-1 circle} (because we use the second convention)
so that the poles, when \emph{transposed} to the other side, are also
outside the circle.
\end{enumerate}
\end{section}
\end{document}
