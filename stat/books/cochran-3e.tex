\documentclass{article}
\usepackage{amsfonts,amsmath,amsthm,url}

\newtheorem{problem}{Problem}
\newenvironment{penum}{
    \renewcommand{\theenumi}{\alph{enumi}}
    \renewcommand{\labelenumi}{(\theenumi)}
    \begin{enumerate}}{\end{enumerate}}


\title{Book Notes \& Exercises: Sampling Techniques 3e}
\author{Cochran, 1977, Wiley}
%\author{Miguel Manese}
\date{}

\begin{document}
\maketitle

\begin{section}{Ch 1: Introduction}

\begin{subsection}{1.3: Principal Steps in a Sample Survey}
\begin{enumerate}
\item Define objectives of the survey.
\item Define population to be sampled.
\item Determine data to be collected -- to be able to do \#1.
\item Set degree of precision desired. This determines the sample size.
\item Determine method of measurement. E.g. via questionnaire vs interview;
  send question by mail / phone / face-to-face.
\item Define the \emph{frame} -- list of \emph{sampling units} that compose
  the population.
  \begin{itemize}
  \item Sampling unit can be ambiguous. E.g. person? family? head of family?
    city? fixed area of land (e.g. 1 hectare, for agricultural survey)?
  \end{itemize}
\item Selection of sample -- balance cost and efficiency (survey design!)
\item Pre-test questionnaire and field methods on small scale -- before
  doing full-fledged survey to see fix unforeseen problems.
\item Do the field work.
\item Analyze data, do summary, answer objectives.
  \begin{itemize}
  \item clean data, analyze, present, interpret.
  \end{itemize}
\item Gather information gained for future survey -- the more you know about 
  the population the better the sampling design (stronger assumptions, etc)
\end{enumerate}
\end{subsection}

\begin{subsection}{1.5: Probability Sampling}
\begin{enumerate}
\item Let $P$ be the population, $S_1, S_2, \ldots S_v \subset P$ be distinct
  samples such that $\cup_i S_i = P$, and $\omega \in S_i \subset P$ for
  one or more $i$ are the \emph{sampling units}.  
\item $P(S_i) = \pi_i$ is the probability of choosing sample $S_i$.
\item The estimator based on $S_i$, $X(S_i)$, is unique for each $i$.
\end{enumerate}
\end{subsection}

\begin{subsection}{1.6: Non-probability Samplint}
Examples:
\begin{enumerate}
\item Sample restricted to readily accessible part of the population (due
  to cost constraint).
\item Sample is made haphazardly -- no planning/design, may be \emph{biased}.
\item Select \emph{typical} units -- subjective bias of sampler
\item Sample consists of volunteers only.
\end{enumerate}
\end{subsection}
\end{section}

\begin{section}{Ch 2: Simple Random Sampling}
\end{section}

\begin{section}{Ch 3: Sampling Proportions and Percentages}
\end{section}


\begin{section}{Ch 4: The Estimation of Sample Size}
\end{section}

\begin{section}{Ch 5: Stratified Random Sampling}
\begin{subsection}{Unsorted}
\begin{enumerate}
\item Partition population into non-overlapping subpopulations 
  $N_1, N_2, \ldots, N_s$. I.e. $N_i \cap N_j = \null$ whenever $i \neq j$,
  and $\cup_i N_i = P$.
\item Choose $n_i$ samples per subpopulation $N_i$.
\end{enumerate}
\end{subsection}
\end{section}

\begin{section}{Ch 5A: Further Aspects of Stratified Sampling}
\end{section}

\begin{section}{Ch 6: Ratio Estimators}
\end{section}

\begin{section}{Ch 7: Regression Estimators}
\end{section}

\begin{section}{Ch 8: Systematic Sampling}
\end{section}

\begin{section}{Ch 9: Signle-stage Cluster Sampling:\\Clusters of Equal Sizes}
\begin{subsection}{Unsorted}
\begin{enumerate}
\item Partition population into non-overlapping \emph{clusters}.
\item Sample clusters.
\item Choose sampling units from chosen clusters
\end{enumerate}
\end{subsection}
\end{section}

\begin{section}{Ch 9A: Single-stage Cluster Sampling:\\Clusters of Unequal Sizes}
\end{section}

\begin{section}{Ch 10: Subsampling with Units of Equal Size}
\end{section}

\begin{section}{Ch 11: Subsampling with Units of Unequal Sizes}
\end{section}

\begin{section}{Ch 12: Double Sampling}
\end{section}

\begin{section}{Ch 13: Sources of Error in Surveys}
\end{section}

\end{document}
