\documentclass{article}
\usepackage{amsfonts,amsmath,amsthm}
\usepackage{multirow}

\title{Unsorted Statistical Tests}
\author{Miguel Manese}
\date{October 13, 2012}

\begin{document}
\maketitle

\begin{section}{Unsorted Tests}
\begin{subsection}{$\chi^2$ Goodness of Fit Test}
Compares observed frequency of occurrence vs expected frequency of occurrence.
Let $X$ be a (discrete) random variable, $X:\Omega \to \Gamma$, $F(X)$ is the
unknown distribution of $X$, $F_0(X)$ is the empirical distribution of $X$,
$O_j = \#\{X_i = j\}$ with $j \in \Gamma$ and $i = 1, \ldots, N$, 
$E_j =$ expected frequency of occurrence of $j$. Test $H_0: F(x) = F_0(x)$ vs
$H_1: F(x) \neq F_0(x)$. Under $H_0$

\[ \hat{\chi}^2 = \sum_{j \in \Gamma} \frac{(O_j - E_j)^2}{E_j} \sim
    \chi^2_\nu \]

\noindent where $\nu = |\Gamma|-1$. Reject $H_0$ at level $\alpha$ if 
$\hat{\chi}^2 > \chi^2_{\nu,\alpha}$.
\end{subsection}

\begin{subsection}{Pearson's $\chi^2$-test}
Given a contingency table, i.e. two factors and the counts for each combination
of levels of those factors, $H_0$: the two factors are independent vs $H_1$:
they are independent. Let $X$ be the random variable for the first factor,
$Y$ for the second factor. If the two factors are independent then the 
expected frequency for cell $\{i,j\}$ is $N \times P(X = i, Y = j) = N \times 
P(X = i) P(Y = j)$. Estimator for $P(X = i)$ is the marginal frequency
for $X = i$.\\

\noindent Example from R's \texttt{chisq.test}:

\begin{table}[h]
\centering
\begin{tabular}{|l|r|r|r|}
\hline
\multirow{2}{*}{Gender} & \multicolumn{3}{|c|}{Party} \\  \cline{2-4}
 & \multicolumn{1}{|c|}{Democrat} &
   \multicolumn{1}{|c|}{Independent} &
   \multicolumn{1}{|c|}{Republican} \\ \hline
Male & 762 & 327 & 468 \\ \hline
Female & 484 & 239 & 477 \\ \hline
\end{tabular}
\end{table}

\noindent Example computation for expected value in upper-leftmost cell
\begin{align}
N &= 762 + 327 + 468 + 484 + 239 + 477 = 2757 & \text{(total observations)} 
\notag\\ 
P(\mathrm{Gender} = \mathrm{Male}) &= \frac{762+327+468}{2757} = 0.5647443 &
\notag\\
P(\mathrm{Party}= \mathrm{Democrat}) &= \frac{762+484}{2757} = 0.4519405 
\notag\\
%P(\mathrm{Gender} = \mathrm{Male}, \mathrm{Party} = \mathrm{Democrat}) &= 
%0.7647443 \times 0.4519405 = 0.2552308 &  \notag\\
E_{\mathrm{Male},\mathrm{Democrat}} &= N \times 0.5647443 \times 0.4519405 
= 703.6714 & \text{(expected frequency)}\notag
\end{align}\\

\noindent Use stat in $\chi^2$ goodness of fit test with $\nu = (r-1)(c-1)$.
\end{subsection}
\end{section}

\begin{section}{Econometrics Tests}
Tests made by econometricians

\begin{subsection}{Durbin-Watson Test}
\end{subsection}

\begin{subsection}{Dickey-Fuller Unit-Root Test}
Given the AR(1) model $X_t = \phi_1 X_{t-1} + \varepsilon_t$, testing
for unit-root means $H_0: \phi_1 = 1$. Differencing $X_t$

\[ \delta X_t = X_t - X_{t-1} = (\phi_1 - 1) X_{t-1} + \varepsilon_t \]

\noindent the null is equivalent to $H_0: \delta = (\phi_1 - 1) = 0$. The
test procedure is to regress $\delta X_t$ vs $X_{t-1}$, and the test
statistic is $\hat{\delta} / \hat{\sigma}_\delta$. For some reason
this is not follow the $t$ distribution and a separate Dickey-Fuller 
table exists for different p-values given sample-size. 

\begin{subsubsection}{Augmented Dickey-Fuller Test}
Let $X_t \sim $ AR($p$), $p > 1$, and it is fitted using an AR(1) model,
say $X_t = \mu + \phi_1 X_{t-1} + \nu_t$. Then

\[ \nu_t = \phi_2 X_{t-2} + \ldots + \phi_p X_{t-p} + \varepsilon_t \]

\noindent and the $\nu_t$ are correlated. Regressing with the AR(1) assumption
will give invalid result. Instead

\begin{align}
X_t &= \mu + \sum_{i=1}^p \phi_i X_{t-i} + \varepsilon_t \notag\\
   &= \mu + (\phi_1 + \phi_2) X_{t-1} - (\phi_2) (X_{t-1} - X_{t-2}) +
      \text{???} \notag
\end{align}

\noindent TODO: it will go down to this

\[ \delta X_t = \mu + \beta t + \delta X_{t-1} + \sum_{i=1}^{p-1}
     \delta X_{t-1-i} + \varepsilon_t \]

\noindent Regress the above equation, same DF stat, same critical values.
\end{subsubsection}
\end{subsection}

\begin{subsection}{Phillips-Peron Unit-Root Test}
\end{subsection}

\begin{subsection}{Johansen's Cointegration Test}
\end{subsection}
\end{section}
\end{document}
