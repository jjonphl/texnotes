%%This is a very basic article template.
%%There is just one section and two subsections.
\documentclass{article}
\usepackage{amsbsy,amsmath}
\newtheorem{proposition}{Proposition}
\newcommand{\Fourier}{\boldsymbol{\mathcal{F}}}
\title{Magadia Thesis Notes}
\begin{document}


\section{Concepts}
Stationary processes are characterized by their spectral density function
(SDF). If $\texttt{sdf}\{Y_{n}\} = \texttt{sdf}\{Z_{n}\}$ then both $Y_{n}$ and
$Z_{n}$ are generated by the same process.

\begin{proposition}
Let $\hat{f_{y}}(w^{*})$ and $\hat{f_{z}}(w^{*})$ be the sample SDF estimate at
fixed frequency $w^{*}$, with $E f_{y}(w^{*}) \neq 0$ and $E f_{z}(w^{*}) \neq
0$. Then under the null hypothesis $H_{0}$ that $Y_{n}$ and $Z_{n}$ are
generated by the same process: 

\[ M_{Z,Y}^{(1)}(w^{*}) = \frac{\log |\hat{f}_{z}(w^{*})| - \log
|\hat{f}_{y}(w^{*})|}{\sqrt{2} \frac{M_{n}}{N} \sum \lambda_{r}^{2}} 
  \overset{d}{\rightarrow} \mathrm{N}(0, 1)\]
\end{proposition}


\begin{itemize}
\item SDF is Fourier transform of the autocorrelation function
(i.e. $\Fourier [R(\tau)] = \hat{f}(w)$).
\item Periodogram is the estimate of the SDF based from the sample.
\item Windowing with $\lambda(r)$, $L(w) = \Fourier [\lambda(r)]$, $\Fourier
[x(t) \lambda(t)] = L(w) \hat{f}(w_{0} - w)$ (real-valued so conjugate is equal
to reverse).
\item $\delta w$ determined at p38
\item Does not use SETAR properties much.
\end{itemize}

\subsection{Proposed Test Procedure}
Assume SETAR-nonlinearity of the form

\[ X_n = \sum_{i=1}^{k_j} \phi_{i}^{(j)} X_{n-i} + \sigma_j \epsilon^{(j)}_n \]

\noindent Given observations \textbf{X} = ($X_1$, $X_2$, ..., $X_N$)


\subsection{Limitations}
\begin{enumerate}
\item Estimation of spectral densities, which this test uses, require a large 
number of observations.
\item At the most, non-stationary processes that can be made stationary
by \emph{variance-stabilizing transformations, differencing, or are
piecewise/locally stationary}. This is because stationarity is a necessary 
condition for the existence of the spectrum.
\end{enumerate}

\end{document}
