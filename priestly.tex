\documentclass{article}
\usepackage{amsfonts}

\title{Priestly Notes}
\author{Miguel Manese}
\date{August 31, 2009}

\newcommand{\tpi}{$2\pi$}
\newcommand{\intpi}{\int_{-\pi}^{\pi}}

\begin{document}

\maketitle
\begin{abstract}
Notes for the book Spectral Analysis and Time Series
\end{abstract}

\begin{section}{Fourier Series: $2\pi$-periodic Functions (4.2)}
\begin{enumerate}
\item $\cos t$, $\sin t$, $\cos 2t$, ... - $2\pi$ periodic 
\item $f(t) = \sum_{n=0}^{\infty} (a_{n}\cos(nt) + b_{n}\sin(nt)$ - 
  $2\pi$-periodic because all components are $2\pi$-periodic
    \begin{enumerate}
    \item can we build $2\pi$-periodic functions from linear combination of
    other functions? (NO)
    \item used by Fourier to solve differential equation in heat conduction,
    Bernoulli in vibrating strings
    \item abstract harmonic analysis in functional analysis, spectral
    resolution of unitary operators
    \end{enumerate}
\item \label{fseries} Canonical form: $X(t) = \frac{1}{2}a_{0} + 
  \sum_{n=1}^{\infty} (a_{n} \cos(nt) + b_{n} \sin(nt))$
    \begin{enumerate}
    \item $n = 0$ $\rightarrow$ bias, $n = 1$ $\to$ fundamental (freq),
    $n = 2$ $\to$ 1st harmonic, etc.
    \end{enumerate}
\item \label{ortho1} $\cos nt$, $\sin mt$ are orthogonal in $[-\pi, \pi]$
    \begin{enumerate}
    \item if $n \neq m$
    \begin{eqnarray}
    \intpi \cos(nt)\cos(mt) dt & = & \frac{1}{2} \intpi [\cos((n+m)t) + 
                 \cos((n-m)t)] dt \nonumber \\
     & = & \left[ \frac{1}{2(n+m)} \sin((n+m)t) +
           \frac{1}{2(n-m)} \sin((n-m)t) \right]_{-\pi}^{\pi} \nonumber \\
    & = & 0 \nonumber
    \end{eqnarray}

    \begin{eqnarray}
    \intpi \cos(nt)\sin(mt) dt & = & \frac{1}{2} \intpi [\sin((n+m)t) +
         \sin((n-m)t)] dt \nonumber \\
     & = & \left[ \frac{-1}{2(n+m)} \cos((n+m)t) + 
                  \frac{-1}{2(n-m)} \cos((n-m)t) \right]_{-\pi}^{\pi} \nonumber\\
     & = & \frac{-1}{2(n+m)} (1 - 1) + \frac{-1}{2(n-m)} (1 - 1) \nonumber\\
     & = & 0 \nonumber
     \end{eqnarray}

    \item if $n = m$
    \begin{eqnarray}
    \intpi \cos^{2}(nt) dt & = & \frac{1}{2} \intpi [\cos(2nt) + 2] dt 
                                                    \nonumber \\
    & = & 0 + \left. \frac{t}{2} \right|_{-\pi}^{\pi} \nonumber\\
    & = & \pi \nonumber
    \end{eqnarray}

    \begin{eqnarray}
    \intpi \cos(nt)\sin(nt) dt & = & \frac{1}{2} \intpi \sin(2nt) dt\nonumber\\
     & = & \frac{-1}{4n} \left. \cos(2nt) \right|_{-\pi}^{\pi} \nonumber\\
     & = & \frac{-1}{4n} (1 - 1) \nonumber\\
     & = & 0 \nonumber
    \end{eqnarray}

    \end{enumerate}
\item Using \#\ref{ortho1}, we can derive a formula for coefficients in 
  \#\ref{fseries}
  \begin{eqnarray}
  \intpi X(t) \cos(mt) dt & = & \frac{1}{2} a_{0} \intpi \cos(mt) dt +
      \sum_{n=1}^{\infty} [a_{n} \intpi \cos(nt)\cos(mt) dt + 
                           b_{n} \intpi \sin(nt)\cos(mt) dt ] \nonumber\\
   & = & a_{m} \intpi \cos^{2}(mt) dt \nonumber\\
   & = & a_{m} \pi \nonumber
  \end{eqnarray}
  \noindent which gives us the ff formulas
  \begin{eqnarray}
  a_{m} & = & \frac{1}{\pi} \intpi X(t) \cos(mt) dt \nonumber\\
  b_{m} & = & \frac{1}{\pi} \intpi X(t) \sin(mt) dt \nonumber
  \end{eqnarray}
\end{enumerate}
\end{section}
\end{document}
