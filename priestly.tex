\documentclass{article}
\usepackage{amsfonts,amsmath}

\title{Priestly Notes}
\author{Miguel Manese}
\date{August 31, 2009}

\newcommand{\tpi}{$2\pi$}
\newcommand{\intpi}{\int_{-\pi}^{\pi}}

\begin{document}

\maketitle
\begin{abstract}
Notes for the book Spectral Analysis and Time Series
\end{abstract}

\begin{section}{Fourier Series: $2\pi$-periodic Functions (4.2)}
\begin{enumerate}
\item $\cos t$, $\sin t$, $\cos 2t$, ... - $2\pi$ periodic 
\item $f(t) = \sum_{n=0}^{\infty} (a_{n}\cos(nt) + b_{n}\sin(nt)$ - 
  $2\pi$-periodic because all components are $2\pi$-periodic
    \begin{enumerate}
    \item can we build $2\pi$-periodic functions from linear combination of
    other functions? (NO)
    \item used by Fourier to solve differential equation in heat conduction,
    Bernoulli in vibrating strings
    \item abstract harmonic analysis in functional analysis, spectral
    resolution of unitary operators
    \end{enumerate}
\item \label{fseries} Canonical form: $X(t) = \frac{1}{2}a_{0} + 
  \sum_{n=1}^{\infty} (a_{n} \cos(nt) + b_{n} \sin(nt))$
    \begin{enumerate}
    \item $n = 0$ $\rightarrow$ bias, $n = 1$ $\to$ fundamental (freq),
    $n = 2$ $\to$ 1st harmonic, etc.
    \end{enumerate}
\item \label{ortho1} $\cos nt$, $\sin mt$ are orthogonal in $[-\pi, \pi]$
    \begin{enumerate}
    \item if $n \neq m$
    \begin{eqnarray}
    \intpi \cos(nt)\cos(mt) dt & = & \frac{1}{2} \intpi [\cos((n+m)t) + 
                 \cos((n-m)t)] dt \nonumber \\
     & = & \left[ \frac{1}{2(n+m)} \sin((n+m)t) +
           \frac{1}{2(n-m)} \sin((n-m)t) \right]_{-\pi}^{\pi} \nonumber \\
    & = & 0 \nonumber
    \end{eqnarray}

    \begin{eqnarray}
    \intpi \cos(nt)\sin(mt) dt & = & \frac{1}{2} \intpi [\sin((n+m)t) +
         \sin((n-m)t)] dt \nonumber \\
     & = & \left[ \frac{-1}{2(n+m)} \cos((n+m)t) + 
                  \frac{-1}{2(n-m)} \cos((n-m)t) \right]_{-\pi}^{\pi} \nonumber\\
     & = & \frac{-1}{2(n+m)} (1 - 1) + \frac{-1}{2(n-m)} (1 - 1) \nonumber\\
     & = & 0 \nonumber
     \end{eqnarray}

    \item if $n = m$
    \begin{eqnarray}
    \intpi \cos^{2}(nt) dt & = & \frac{1}{2} \intpi [\cos(2nt) + 2] dt 
                                                    \nonumber \\
    & = & 0 + \left. \frac{t}{2} \right|_{-\pi}^{\pi} \nonumber\\
    & = & \pi \nonumber
    \end{eqnarray}

    \begin{eqnarray}
    \intpi \cos(nt)\sin(nt) dt & = & \frac{1}{2} \intpi \sin(2nt) dt\nonumber\\
     & = & \frac{-1}{4n} \left. \cos(2nt) \right|_{-\pi}^{\pi} \nonumber\\
     & = & \frac{-1}{4n} (1 - 1) \nonumber\\
     & = & 0 \nonumber
    \end{eqnarray}

    \end{enumerate}
\item \label{formula1} Using \#\ref{ortho1}, we can derive a formula for 
  coefficients in \#\ref{fseries}
  \begin{align}
  \intpi X(t) \cos(mt) dt & = \frac{1}{2} a_{0} \intpi \cos(mt) dt +
      \sum_{n=1}^{\infty} [a_{n} \intpi \cos(nt)\cos(mt) dt + 
                           b_{n} \intpi \sin(nt)\cos(mt) dt ] \nonumber\\
   & = a_{m} \intpi \cos^{2}(mt) dt \nonumber\\
   & = a_{m} \pi \nonumber
  \end{align}
  \noindent which gives us the ff formulas
  \begin{align}
  a_{0} & =  \frac{1}{\pi} \intpi X(t) dt  \notag\\
  a_{m} & =  \frac{1}{\pi} \intpi X(t) \cos(mt) dt && \text{m = 1, 2, ...} \notag\\
  b_{m} & =  \frac{1}{\pi} \intpi X(t) \sin(mt) dt && \text{m = 1, 2, ...} \notag
  \end{align}

\item \label{justify} Justifcations for \#\ref{formula1} - difficult questions 
of infinite sums
  \begin{enumerate}
  \item Is term by term integration ok? (i.e. $\int \cos(nt)\cos(mt) dt$?) -
  can any $2\pi$-periodic function really be represented by \#\ref{fseries}?
  \item Conditions: $\intpi |X(t)| dt < \infty$ (absolutely integrable,
  i.e. $X \in \mathcal{L}^{1}(-\pi,\pi)$, which implies the following

  \[ 
    |a_{m}| \leq \intpi |X(t)| |\cos(mt)| dt \leq \intpi |X(t)| dt \leq \infty
  \]

  \item Question set 1:
    \begin{enumerate}
    \item Can any $2\pi$-periodic function be represented as 
    $f(t) = \frac{1}{2}a_{0} + \sum_{n=1}^{\infty}(a_{n} \cos(nt) + 
                                                   b_{n} \sin(nt))$
    \item Term by term integration ok in the expansion of $f(t)$ above? This
    \emph{problem} can be fixed by assuming RHS uniformly converges.
    \end{enumerate}

  \item Question set 2, given $\{a_{n}\}$, $\{b_{n}\}$ defined in 
  \#\ref{formula1}
    \begin{enumerate}
    \item Are they finite? Yes if $\intpi |X(t)| dt < \infty$.
    \item Does it converge?
    \item Does it converge to $X(t)$ ?
    \end{enumerate}
  \end{enumerate}  

\item X(t) is of \emph{bounded variation} in $[a, b]$ if for a fixed $n$,
all subdivisions of $[a, b]$ with $n$ segments 
$(a = t_{0} < t_{1} < ... < t_{n} = b)$, 
$\sum_{v=0}^{n-1} |X(t_{v+1}) - X(t_{v})| < B(n)$. I.e. bounded by some
$B$ dependent only on $n$, not the specific subdivision
  \begin{enumerate}
  \item $X(t)$ is of a bounded variation if it has finite number of maxima,
  minima and discontinuities.
  \item E.g. $\tan(\theta)$ is of bounded variation.
  \end{enumerate}

\item Jordan's test: if $X(t)$ is of a bounded variation in the neighborhood
of $t = t_{0}$ (i.e. some interval around $t_{0}$) then its fourier seris 
converges to $\frac{1}{2}(X(t_{0}-0) + X_(t_{0}+0))$

\item In \#\ref{justify} we only assumed that $X \in \mathcal{L}^{1}(-\pi,\pi)$.
Here we assume $X \in \mathcal{L}^{2}(-\pi,\pi)$. Then if 
$X_{m}(t) = \sum_{n=0}^{m}(a_{n}\cos(nt) + b_{n}\sin(nt))$, 
$X_{m}(t) \rightarrow X(t)$ in the \emph{mean square} sense 
($\intpi [X(t) - X_{m}(t)]^{2} dt \rightarrow 0$ as $m \rightarrow \infty$)
\end{enumerate}
\end{section}

\begin{section}{Fourier Series of General Periodicity (4.3)}
\begin{enumerate}
\item Let $X(t)$ be $2T$-periodic, $Y(t) \equiv X(\frac{T}{\pi} t)$, then
  \begin{align}
  Y(t+2\pi) & = X\left(\frac{(t + 2\pi)T}{\pi}\right) \notag\\
   & = X\left(\frac{tT}{\pi} + \frac{2 \pi T}{\pi}\right) \notag\\
   & = X\left(\frac{tT}{\pi} + 2T\right) \notag\\
   & = X\left(\frac{tT}{\pi}\right) = Y(t) \notag
  \end{align} 
  \noindent i.e. Y(t) is $2\pi$ periodic

\item Using previous results for $2\pi$ periodic functions, 
$Y(s) = \frac{1}{2}a_{0} + \sum_{n=1}^{\infty}(\ldots)$. 
Let $s = \frac{\pi t}{T}$, then 
$X(t) = Y(\frac{\pi}{T} t) = \frac{1}{2}a_{0} + \sum_{n=1}^{\infty}
 (a_{n}cos(\frac{\pi t}{T} n) + b_{n}sin(\frac{\pi t}{T} n))$

\item $\cos(\frac{\pi t}{T}n)$ and $\sin(\frac{\pi t}{T}n)$ are also orthognal
\end{enumerate}
\end{section}

\begin{section}{Spectral Analysis of Periodic Functions (4.4)}

\end{section}
\end{document}
