\documentclass{article}
\usepackage{amsfonts, amsmath}

\title{Math Jargons}
\author{Miguel Manese}


\begin{document}
\maketitle

\begin{section}{Analysis}
\begin{subsection}{Sets and Classes}
\begin{enumerate}
\item \emph{Denumerable} -- old term for \emph{countable}
\item \emph{Group}
\item \emph{Field}
\item \emph{Algebra}
\item \emph{Ring}
\item \emph{Semi-ring}
\item \emph{$\sigma$-algebra}
\end{enumerate}

\end{subsection}

\begin{subsection}{Mappings and Functions}
\begin{enumerate}
\item General maps $f : X \to Y$
  \begin{enumerate}
  \item \emph{Injective} (one is to one) -- $f(x_1) = f(x_2)$ iff $x_1 = x_2$
  \item \emph{Surjective} (onto) -- $\forall y \in Y \exists x \in X$ s.t. 
        $y = f(x)$
  \item \emph{Bijective} (one is to one and onto) -- both injective and 
        surjective
  \end{enumerate}
\item \emph{Homomorphisms} -- linear maps $f : X \to Y$
  \begin{enumerate}
  \item \emph{Monomorphism} - injective
  \item \emph{Epimorphism} - surjective
  \item \emph{Isomorphism} - bijective
  \item \emph{Endomorphism} - $X = Y$, i.e. $f$ is a mapping from $X$ to itself
  \item \emph{Automorphism} - bijective and $X = Y$
  \end{enumerate}
\item \emph{Homeomorphism} - a continuous deformation with a continuous 
      inverse. (related to being \emph{topologically equivalent})
\item \emph{Homogeneous functions} -- if $f(x) = y$ implies $f(kx) = ky$
\end{enumerate}
\end{subsection}

\end{section}

\begin{section}{Calculus}
\begin{subsection}{Vector Calculus}
\begin{enumerate}
\item Derivatives for $f : \Re^n \to \Re$, where $n > 1$
    \begin{enumerate}
    \item \emph{Gradient} - vector of first derivatives
    \[ \nabla f(\mathbf{x}) = \left[ \frac{\partial}{\partial x_i} \right] 
           f(\mathbf{x}) \]
    \item \emph{Hessian} - matrix of "pure" and "mixed" second derivatives
          \[ \nabla^2 f(\mathbf{x}) = 
            \left[\frac{\partial^2}{\partial x_{i} \partial x_{j}}\right] 
            f(\mathbf{x}) \]
    \item \emph{Laplacian} - scalar sum of "unmixed" second derivatives
    \end{enumerate}
\item Derivatives for $\mathbf{f} : \Re^n \to \Re^m$, where $m, n > 1$. Let 
      $f_1, \ldots, f_m : \Re \to \Re$ be the scalar-valued components of 
      $\mathbf{f}$, i.e.
      $\mathbf{f}(\mathbf{x}) = [f_1(\mathbf{x}) f_2(\mathbf{x}) \ldots
        f_m(\mathbf{x})]'$
    \begin{enumerate}
    \item \emph{Jacobian} - matrix where $i$th row is the gradient of the 
          $i$th function
          \[ J(f) = \begin{bmatrix}
                    \nabla f_1(\mathbf{x}) \\
                    \nabla f_2(\mathbf{x}) \\
                    \vdots \\
                    \nabla f_m(\mathbf{x})
                    \end{bmatrix} \]
    \item \emph{Divergence} - scalar sum where the $i$th term is the 
          derivative of the $i$th function with respect to the $i$th variable
    \end{enumerate}
\end{enumerate}
\end{subsection}
\end{section}

\begin{section}{Complex Analysis}
\begin{enumerate}
\item \emph{Cauchy-Riemann equations} -- given $f : \mathbb{C} \to \mathbb{C}$
      defined by $f(x + iy) = u(x,y) + i v(x,y)$ where $u, v: \Re \to \Re$,
      let $u_x = \frac{\partial}{\partial x} u$. If $f'(z_0)$ exists then
      it must satisfy the following equations (Cauchy-Riemann equations)
      \begin{align}
      u_x(x_0, y_0) &= v_y(x_0, y_0) \notag \\
      u_y(x_0, y_0) &= - v_x(x_0, y_0) \notag
      \end{align}
\item \emph{Analytic} -- $f : \mathbb{C} \to \mathbb{C}$ is analytic on a
      point $z_0$ or an open set $S \subset \mathbb{C}$ if it is differentiable
      at $z_0$ or on all $z \in S$, respectively.

\end{enumerate}
\end{section}

\begin{section}{Linear Operators}
\noindent Linear maps and matrices

\begin{enumerate}
\item Hermitian matrices -- generalization of symmetric matrices over $\Re$ to
      matrices over $\mathbb{C}$. $A$ is hermitian if 
\[ A = \overline{A'} \]
\item Unitary matrices -- generalization of orthognal matrices over $\Re$ to
      matrices over $\mathbb{C}$. $A$ is unitary if
\[ A \overline{A'} = I \]
\end{enumerate}
\end{section}

\begin{section}{Dynamical Systems}
\begin{enumerate}
\item \emph{Trajectory}, \emph{orbit} -- path of solution to $x_{t+1} = f(x_t)$
\item \emph{Absorbing Set} --
\item \emph{Attractor} --
\item \emph{Natural measure} -- frequency of points visited by trajectory,
      associated with a specific trajectory
\item \emph{Ergodicity} -- (heuristic) average over trajectory converges to
      ensemble marginal average (specific random variable $X[t]$)
\item \emph{Stationary distribution} -- $\tau$ is stationary if whenever
      $x_0 \sim \tau \to x_1 \sim \tau$. I.e. if $x_0 \sim \tau$, marginal
      distribution for all indices $t > 0$ are also $\tau$
\item \emph{Lyapunov exponent} -- 
\item \emph{Trapping region} --
\item \emph{Correlation dimension} --
\item \emph{Embedding dimension} --
\item \emph{Chaotic} -- iff Lyapunov exponent $> 1$
\item \emph{Homoclinic orbits} -- trajectories that start and end at the same
      fixed point
\item \emph{Nullclines} - curves in the phase portrait where a component 
      derivative is 0, e.g. in 2-dimensions the continuous curve where
      $\dot{x} = 0$.
\item \emph{Basin of attraction} - for a fixed point $x^*$ are all points
      $x$ such that all trajectories starting from such $x$ have $x_t \to x^*$
\item Types of stability
      \begin{enumerate}
      \item \emph{Lagrange stability} - $\forall x \in \Omega$, orbit starting
      from $x$ is a precompact subset of $\Omega$, i.e. $|x(t)|$ is bounded
      for all $t$.
      \item \emph{Lyapunov stability} - trajectories of close points remain
      close. Formally, for discrete time, $\forall x, y \in \Omega$
      \[ \lim_{y \to x} \sup_{k \geq 0} \| T^k y - T^k x \| = 0 \]
      \item \emph{Asymptotic stability} or \emph{attracting fixed point} - 
      all trajectory starting near fixed point $x^*$ approaches/converges
      to $x^*$
      \item \emph{Global asymptotic stability} or \emph{globally attracting
      fixed point} - Lyapunov stable and all starting point converges to
      some fixed point $x^*$
      \end{enumerate}
\end{enumerate}
\end{section}

\begin{section}{Measure Theory and Probability}
\begin{subsection}{Measure Theory}
\begin{enumerate}
\item 
\end{enumerate}
\end{subsection}
\end{section}

\begin{section}{Stochastic Processes}
\begin{subsection}{Processes}

\end{subsection}
\end{section}
\end{document}
