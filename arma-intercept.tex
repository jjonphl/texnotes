\documentclass{article}
\usepackage{amsfonts,amsmath,amsthm,url}

\title{ARMA Intercept Term}
\author{Miguel Manese}
\date{Aug. 21, 2009}

\newtheorem{definition}{Definition}
\newtheorem{example}{Example}
\newcommand{\Xt}{\{X_{t}\}}
\newcommand{\Yt}{\{Y_{t}\}}
\newcommand{\E}{\mathrm{E}}
\begin{document}
\maketitle

\noindent Let $\Xt$ be a (weakly) stationary and invertible time series.

\begin{definition}\normalfont
$\Xt$ is MA(p) if $X_{t} = \sum_{j=0}^{p} \theta_{j} Z_{t-j}$, where 
$\theta_{0} = 1$ and $Z_{t} \sim \mathrm{WN}(0, \sigma^2)$.
\end{definition}

\noindent 
If $\Xt$ is MA(p), $\E X_{t} = \sum_{j=0}^{p} \theta_{j} \E Z_{t-j} = 0$. 
Let $\Yt = \Xt + \alpha$, then $\E Y_{t} = \alpha$ and $\alpha$ is also the
\textit{intercept} of the series.


\begin{definition}\normalfont
$\Xt$ is AR(p) if $\sum_{j=0}^{p} \phi_{j} X_{t-j} = Z_{t}$, where 
$\phi_{0} = 1$ and $Z_{t} \sim \mathrm{WN}(0, \sigma^2)$.
\end{definition}

\noindent
If $\Yt$ is AR(p), then it can be expressed as an MA($\infty$) process
$X_{t} = \sum_{j=0}^{\infty} \psi_{j} Z_{t-j}$. In the
latter representation it is easy to see that $\E X_{t} = 0$. An intercept term
does not easily fit an intercept term, as shown below.

\begin{example}\label{ex:ar_alpha}\normalfont
Let $\Xt$ be a time series with $X_{t} - 0.2 X_{t-1} = 1 + Z_{t}$. This is 
(like) an \textit{AR(1) with intercept $\alpha = 1$}. To compute $\E X_{t}$

\begin{eqnarray}
X_{t} & = & 1 + 0.2 X_{t-1} + Z_{t} \nonumber\\
         & = & 1 + 0.2 [1 + 0.2 X_{t-2} + Z_{t-1}] + Z_{t} \nonumber\\
         & = & 1 + 0.2 + 0.04 X_{t-2} + 0.2 Z_{t-1} + Z_{t} \nonumber\\
         & = & 1.2 + 0.04 [1 + 0.2 X_{t-3} + Z_{t-2}] + 0.2 Z_{t-1} + Z_{t} \nonumber\\
         & = & 1.24 + 0.008 X_{t-3} + 0.04 Z_{t-2} + 0.2 Z_{t-1} + Z_{t} \nonumber\\
         & = & \ldots \nonumber
\end{eqnarray}

\noindent And so 

\[
\E X_{t}  =  \alpha (1 + \sum_{j=1}^{\infty} \phi^{j} ) 
   =  \alpha \sum_{j=0}^{\infty} \phi^{j} 
   =  \frac{\alpha}{1 - \phi} \nonumber 
   =  \frac{1}{1 - 0.2} = 1.25
\]
\end{example}

\noindent \qedsymbol\\

\noindent We see that $\E X_{t} \neq \alpha$, as in the case for MA(p)
with added intercept. Conversely the following example shows how to get the 
\textit{intercept} given $\E X_{t}$ of Example~\ref{ex:ar_alpha}

\begin{example}\label{ex:ar_mean}\normalfont
Let $\Xt$ be an AR(1) process with $\E X_{t} = 1.25$. Then we have
$(X_{t} - 1.25) - 0.2(X_{t-1} - 1.25) = Z_{t}$. Rearranging terms to get
the intercept $\alpha$

\begin{eqnarray}
X_{t} & = & 1.25 - 0.2(X_{t-1} - 1.25) + Z_{t} \nonumber\\
  & = & 1.25 - 0.2 \cdot 1.25 - 0.2 X_{t-1} + Z_{t} \nonumber\\
  & = & 1 - 0.2 X_{t-1} + Z_{t} \nonumber
\end{eqnarray}

\noindent which shows that the intercept $\alpha=1$. \qedsymbol 
\end{example}

\noindent Relationship between the intercept term and the mean of an AR(p)
process is more complicated for $p > 1$. (This should be derivable, though.)

\begin{definition}\normalfont
$\Xt$ is ARMA(p, q) if $\sum_{j=0}^{p} \phi_{j} X_{t-j} = 
\sum_{j=0}^{q} \theta_{j} Z_{t-j}$, where 
$\phi_{0} = \theta_{0} = 1$ and $Z_{t} \sim \mathrm{WN}(0, \sigma^2)$.
\end{definition}

\noindent Intercept \textit{issues} for ARMA(p, q) is similar to those of
AR(p).

\begin{thebibliography}{9}
\bibitem{bRProb} Shumway and Stoffer,
\emph{Some R Time Series Issues},  
\url{http://www.stat.pitt.edu/stoffer/tsa2/Rissues.htm},
retrieved August 31, 2009.
\end{thebibliography}
\end{document}
