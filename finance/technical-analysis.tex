\documentclass{article}
\usepackage{amsfonts, amsmath}

\title{Technical Analysis Notes}
\author{Miguel Manese}
\date{}

\begin{document}
\maketitle

\begin{section}{Technical Indicators}
\begin{subsection}{Classifications}

\begin{enumerate}
  \item Leading indicators -- leads price movements, i.e. they anticipate price movements.
  \item Lagging indicators -- past values greatly influence current value for the indicator
\end{enumerate}

\begin{enumerate}
\item Trend following indicators - 
\item Momentum indicators - 
\item ...
\end{enumerate}

\end{subsection}

\begin{subsection}{Moving Averages}

  \begin{subsubsection}{Simple Moving Average}
    Let $n$ be the moving average \emph{lag}.

    \[ \text{SMA} = \frac{1}{n} \sum_{t=1}^n P_t \]
  \end{subsubsection}

  \begin{subsubsection}{Exponential Moving Average}
    \begin{align}
      \text{EMA}_t &= K P_t + (1 - K) \text{EMA}_{t-1} \notag\\
                   &= \text{EMA}_{t-1} + K (P_t - \text{EMA}_{t-1}) \notag
    \end{align}
  \end{subsubsection}


  \begin{subsubsection}{Weighted Moving Average}
    \[ \text{WMA}_t = \frac{1}{n!} \sum_{i=0}^{n-1} P_{t-i} \times i \]
  \end{subsubsection}

  \begin{subsubsection}{Welles Wilder Moving Average}
    Like EMA but $K = \frac{1}{n}$
    \[ \text{WWMA}_t = K P_t + (1 - K) \text{WMMA}_{t-1} \]

    (Boundary condition) if lag is $n$, for compute $n-1$ as
    \[ \text{WMMA}_{n-1} = \frac{1}{n-1} \sum_{t=1}^{n-1} P_t \]
    then use the formula above
  \end{subsubsection}

  \begin{subsubsection}{Double Exponential Moving Average}
    \[ \text{DEMA}_t = 2 \times \text{EMA}(\{P_t\}) - 
       \text{EMA}(\text{EMA}(\{P_t\})) \]
  \end{subsubsection}


  \begin{subsubsection}{Generalized DEMA}
    Let $v \in [0, 1]$ (v-factor). $v = 1$ is DEMA

    \[ \text{GDEMA}_t = (1 + v) \times \text{EMA}(\{P_t\}) - 
            v \times \text{EMA}(\text{EMA}(\{P_t\})) \]
  \end{subsubsection}

  \begin{subsubsection}{T3}
    Uses GDEMA with $v$ usually set at 0.7

    \[ \text{T3} = \text{GDEMA}(\text{GDEMA}(\text{GDEMA}(\{P_t\}))) \]
  \end{subsubsection}

  \begin{subsubsection}{Elastic Volume-Weighted Moving Average}
    Let $n =$ \textit{volume period}

    \[ \text{EVWMA}_t = \frac{1}{n} [(n - V_t) \text{EVWMA}_{t-1} + V_t P_t] 
         \text{(???)}\]
  \end{subsubsection}

  \begin{subsubsection}{Zero-Lag Exponential Moving Average}
    --
  \end{subsubsection}

  \begin{subsubsection}{Volume-weighted Average Price}
    --
  \end{subsubsection}

  \begin{subsubsection}{Volume-weighted Moving Average}
  \end{subsubsection}

  \begin{subsubsection}{Variable Moving Average}
  \end{subsubsection}


\end{subsection}

\begin{subsection}{Unsorted Indicators}


\begin{description}
\item[Moving Average Convergence-Divergence (MACD)] momentum oscillator
  \begin{align}
  \text{MACD}_t &= \text{EMA}^{12}_t(\{P_t\}) - \text{EMA}^{26}_t(\{P_t\}) \notag\\
  \text{MACD}^{\text{signal}}_t &= \text{EMA}^9_t(\text{MACD}_t)\notag\\
  \text{MACD}^{\text{histogram}}_t &= \text{MACD}_t - \text{MACD}^{\text{signal}}_t \notag
  \end{align}

\noindent When $\text{MACD}_t \nearrow 0$ then the shorter EMA is increasing
and/or longer EMA is decreasing which means upside momentum is increasing.
Similarly $\text{MACD}_t \searrow 0$ means downside momentum is increasing.

\item[Relative Strength Index (RSI)] 
\item[Commodity Channel Index (CCI)]
\item[Stochastic Oscillator]
\item[Williams \%R]
\end{description}
\end{subsection}

\begin{subsection}{Lagging Indicators}

\end{subsection}
\end{section}
\end{document}
