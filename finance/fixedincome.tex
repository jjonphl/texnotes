\documentclass{article}
\usepackage{amsfonts, amsmath}

\newcommand{\CF}{C\!F}

\title{Fixed Income Notes}
\author{Miguel Manese}
\date{}

\begin{document}
\maketitle

\begin{section}{Duration and Convexity}
Let $\CF_t$ be the cashflow at time $t$, $P$ the price per 100 par, $y$ the
yield, and $f(\cdot)$ the function that computes the price $P$ from yield
$y$. Then

\begin{align}
P_0 &= f(y_0) \notag\\
  &= \sum_{i=1}^N \frac{\CF_{t_{i}}}{(1 + y_0)^{t_{i}}} \notag
\end{align}

The duration and convexity are sensitivity measures related to the first
and second derivative of $f(\cdot)$ respectively. The first derivative of
$f(\cdot)$ can be expressed as

\begin{align}
f'(y) &= \sum_{i=1}^N \frac{(- t_i) \CF_{t_{i}}}{(1 + y)^{t_i + 1}} \notag\\
      &= - \left[\left(\sum_{i=1} \frac{t_i \CF_{t_{i}}}{(1 + y)^{t_i}} \right)
        \times \frac{1}{y + i} \times \frac{1}{P}\right] \times P \notag\\
      &= - \frac{D^*}{1 + y} \times P \notag\\
      &= - D \times P \notag
\end{align}

\noindent where $D^{*}$ and $D$ are the \emph{Macaulay duration} and 
\emph{modified duration} respectively. The second derivative can be expressed
as

\begin{align}
f''(y) &= \sum_{i=1}^N \frac{(- t_i) (-1)(t_i + 1) 
   \CF_{t_{i}}}{(1 + y)^{t_i + 2}} \notag\\
      &= \left[\left(\sum_{i=1}^N \frac{t_i (t_i + 1) \CF_{t_{i}}}{(1 + y)^{t_i}}
        \right) \times \frac{1}{(1 + y)^2} \times \frac{1}{P}\right]
        \times P \notag\\
      &= C \times P \notag
\end{align}

\noindent where $C$ is the \emph{convexity}. In terms of $D$ and $C$, the
second-order Taylor-series expansion of $f(\cdot)$ about $y_0$ is

\begin{align}
P = f(y) &\approx P_0 + f'(y_0) (y - y_0) + \frac{1}{2} f''(y_0) (y - y_0)^2
  \notag\\
     &= P_0 + (-1) D \times P_0 (\Delta y) + 
        \frac{1}{2} C \times P_0 (\Delta y)^2 \notag\\
\Delta P &= - D \times P_0 (\Delta y) + \frac{1}{2} C \times P_0 (\Delta y)^2
  \notag
\end{align}

In the above formulas $t_i$ are assumed to be year-fractions and that
interest are compounded annually. For RPI with coupon periods less than
one year, use $t_i$ that are in terms of coupon period and annualize
the resulting $D$ and $C$ by $\frac{1}{2}$ and $\frac{1}{2^2} = \frac{1}{4}$
respectively.

The \emph{effective duration} and \emph{effective convexity} are based on
the numerical derivative of $f(\cdot)$ about $y_0$. Given a small $\Delta y$,
let $P_{\mp} = f(y_0 \mp \Delta y)$ and $D_{\mp} = $ the duration at 
$y_0 \mp \Delta y$

\begin{align}
D_E &= \frac{P_{-} - P_{+}}{2 P_0 \Delta y} \notag\\
C_E &= \frac{D_{-} - D_{+}}{\Delta y} \notag\\
    &= \left(\frac{f(y_0 - \Delta y) - P_0}{P_0 \Delta y} - 
              \frac{P_0 - f(y_0 + \Delta y)}{P_0 \Delta y}\right)
       \times \frac{1}{\Delta y}\notag
\end{align}
\end{section}

\end{document}
