\documentclass{article}
\usepackage{amsfonts, amsmath}

\title{Difference Equations}
\author{Miguel Manese}
\date{November 11, 2009}

\newcommand{\E}{\mathrm{E}}
\newcommand{\Like}{\mathrm{L}}
\newcommand{\Var}{\mathrm{Var}}
\newcommand{\bo}[1]{\boldsymbol{#1}}
\newtheorem{definition}{Definition}
\begin{document}
\maketitle

\begin{section}{Difference Equations with Constant Coefficients}
\begin{enumerate}
\item General solution
  \begin{align}
  y(n) &= y_h(n) + y_p(n) && \text{homegeneous + particular}\notag
  \end{align}
\item Solving for the homegeneous solution
  \begin{enumerate}
  \item Set the \emph{forcing function} to 0: $\sum_{k=0}^N a_k y(n-k) = 0$
  \item Assume solution is $y_h(n) = \lambda^n$, then
   \[ \sum_{k=0}^N a_k \lambda^{n-k} = \lambda^{n-N} 
      (\lambda^N + a_1 \lambda^{N-1} + \ldots + a_{N-1} \lambda + a_N) = 0 \]
  \noindnet which is called the \emph{characteristic polynomial}
  \item The roots are the solutions
  \item Assume the solutions $\lambda_1$, $\lambda_2$, $\ldots$, $\lambda_N$
  are unique, then
   \[ y_h(n) = C_1\lambda_1^n + C_2\lambda_2^n + ... + C_N\lambda_N^n \]
  \item Also known as the \emph{zero input response}  (??)
  \end{enumerate}
\end{enumerate}

Unsorted stuffs
\begin{enumerate}
\item Zero-input - depends on initial condition & system (natural response).
\item Zero-state - depends on input (initial condition is 0)
\end{section}
\end{document}
