\documentclass{article}
\usepackage{amsfonts, amsmath, amsthm}

\newcommand{\Int}{\mathcal{Z}}
\newcommand{\Cplx}{\mathcal{C}}

\title{Unsorted Notes}
\author{Miguel Manese}
\date{}

\begin{document}
\maketitle

\begin{section}{Time Series 2013-03-22}
\begin{enumerate}
\item Branching process
  \[ X_{t+1} = \sum_{j=1}^{X_t} Z_{t,j} \]
  where $\{Z_{t,j}\}$ are IID of $\Int^+$.
\item Kolmogorov's Theorem: Let 
  $\Gamma = \{\mathbf{t} = (t_1, \ldots, t_n)' \in T^n :
     t_1 < t_2 < \ldots < t_n, n = 1, \ldots \}$, $T \subset \Re$,
  $F_{\mathbf{t}} = P(X_{t_1} \leq x_1, \ldots, X_{t_n} \leq x_n)$,
  $\mathbf{x} \in \Re^n$. The probability distribution function
  $\{F_{\mathbf{t}}(\cdot), \mathbf{t} \in \Gamma\}$ are the df of some
  stochastic process iff $\forall n \in \Int^+$, 
  $\mathbf{t} \in \Gamma$ and $1 \leq i \leq n$, 
  $\lim_{x_i \to \infty} F_{\mathbf{t}}(\mathbf{x}) = 
     F_{\mathbf{t}_{(i)}}(\mathbf{x}_{(i)})$ where $\mathbf{t}_{(i)}$ and
  $\mathbf{x}_{(i)}$ are the $(n-1)$-component vectors obtained by deleting
  the $i$-th component of $\mathbf{t}$ and $\mathbf{x}$.\\

  If $\phi_{\mathbf{t}}(\cdot)$ is the characteristic function for 
  $F_{\mathbf{t}}(\cdot)$, the theorem is equivalent to
  $\lim_{u_i \to 0} \phi_{\mathbf{t}}(\mathbf{u}) = 
    \phi_{\mathbf{t}_{(i)}}(\mathbf{u}_{(i)})$ where 
  \[\phi_{\mathbf{t}}(\mathbf{u}) = \int_{\Re^n} \exp(i \mathbf{u}'\mathbf{x})
     F_{\mathbf{t}}(dx_1, \ldots, dx_2)\]
\item Autocovariance functions -- for (weakly) stationary processes
  \begin{enumerate}
  \item If $\gamma(\cdot)$ is the ACVF of stationary proc $\{X_t\}$ then
    \begin{enumerate}
    \item $\gamma(0) \geq 0$
    \item $|\gamma(h)| \leq \gamma(0) \forall \Int$
    \item $\gamma(h) = \gamma(-h) \forall h \in \Int$
    \end{enumerate}
  \item Theorem 1.5.1 (p27, characterization of ACVF) 
  \label{brockwell:thm:1.5.1}
  $\gamma : \Int \to \Re$ is the ACVF of a stationary process iff
  it is even ($\gamma(h) = \gamma(-h)$) and non-negative definite
  ($\forall n \in \Int^+, \mathbf{a} \in \Re^n, \mathbf{t} \in \Int^n 
   \to \sum_{i,j=1}^n a_i \gamma(t_i - t_j) a_j \geq 0$)
  \end{enumerate}
\end{enumerate}
\end{section}

\begin{section}{Time Series, Brockwell 20120322}
\begin{enumerate}
\item Theorem 1.5.1 (p27, characterization of ACVF) -- see 
  \ref{brockwell:thm:1.5.1}.
\item Theorem 4.3.1 (p117, Herglotz') -- $\gamma : \Int \to \Cplx$ is 
  non-negative definite iff $\exists F : [-\pi,\pi] \to \Re^+$ which
  is right-continuous, non-decreasing and bounded with $F(-\pi) = 0$ 
  such that $\gamma(h) = \int_{(-\pi,\pi]} e^{ihv} dF(v)$ $\forall h \in \Int$. 
  \begin{enumerate}
  \item $F(\cdot)$ is the \emph{spectral distribution} of $\gamma$
  \item $f = dF$ is the \emph{spectral density} of $\gamma$ (if it exists)
  \item $\gamma$ is Hermitian ($\gamma(-h) = \overline{\gamma(h)}$,
    complex equivalent of even/symmetric) if 
    $\gamma(h) = \int_{(-\pi,pi]} e^{ihv} dF(v)$
  \item $\gamma(h)$ non-negative definite $\to \gamma(h)$ is Hermitian
  \end{enumerate}
\item Corollary 4.3.1 (p119) $\gamma: \Int \to \Re$ is the ACVF of a
  stationary process iff either
  \begin{itemize}
  \item $\gamma(h) = \int_{(-\pi,\pi]} e^{ihv} dF(v)$ $h \in \Int$, 
  $F$ is right continuous, non-decreasing, bounded and $F(-\pi) = 0$. Or,
  \item $\sum_{i=1}^N \sum_{j=1}^N a_i \gamma(i-j) \overline{a_j} \geq 0$
  $\forall N \in \Int^+$, $a \in \Cplx^N$.
  \end{itemize}
\end{enumerate}
\end{section}

\begin{section}{Characteristic Funciton for Standard Normal 20130407}
The standard normal pdf is given by

\[f(x) = \frac{1}{\sqrt{2\pi}} \int_{-\infty}^\infty e^{\frac{-x^2}{2}} dx \]

\begin{enumerate}
\item \emph{Physicists' version}

\begin{align}
\varphi(t) &= E[e^{itX}] = \int_{-\infty}^\infty e^{itx} \frac{1}{\sqrt{2\pi}}
   e^{\frac{-x^2}{2}} dx \notag\\
           &= \frac{1}{\sqrt{2\pi}} \int_{-\infty}^{\infty} 
               \exp\left(-\frac{x^2}{2} + itx\right) dx \notag\\
           &= \frac{1}{\sqrt{2\pi}} \int_{-\infty}^{\infty} 
               \exp\left(\frac{1}{2} (-x^2 + 2itx - (it)^2 + (it)^2)\right) 
               dx \notag\\
           &= \frac{e^{\frac{(it)^2}{2}}}{\sqrt{2\pi}} \int_{-\infty}^{\infty} 
               \exp\left(-\frac{1}{2} (x^2 - 2itx + (it)^2)\right) dx \notag\\
           &= \frac{e^{-\frac{t^2}{2}}}{\sqrt{2\pi}} \int_{-\infty}^{\infty} 
               \exp\left(-\frac{1}{2} (x + it)^2\right) dx \notag\\
           &= e^{-\frac{t^2}{2}} \cdot 1 
             \qquad\text{Normal with mean $-it$ and variance 1} \notag
\end{align}
\item \emph{Mathematician's version}

\begin{align}
\varphi(t) &= \int_{-\infty}^\infty e^{itx} \frac{1}{\sqrt{2\pi}}
              e^{-\frac{x^2}{2}} dx \notag\\
           &= \int_{-\infty}^\infty (\cos(tx) + i\sin(tx)) \frac{1}{\sqrt{2\pi}}
              e^{-\frac{x^2}{2}} dx \notag\\
           &= \frac{1}{\sqrt{2\pi}} \int_{-\infty}^\infty \cos(tx) 
               e^{-\frac{x^2}{2}} dx +
              \frac{i}{\sqrt{2\pi}} \int_{-\infty}^\infty \sin(tx) 
               e^{-\frac{x^2}{2}} dx \notag
\end{align}

Since $\sin(-x) = -\sin(x)$

\begin{align}
\frac{i}{\sqrt{2\pi}} \int_{-\infty}^\infty \sin(tx) e^{-\frac{x^2}{2}} dx 
    &= \frac{i}{\sqrt{2\pi}} \int_{-\infty}^0 \sin(tx) e^{-\frac{x^2}{2}} dx +
       \frac{i}{\sqrt{2\pi}} \int_{0}^\infty \sin(tx) e^{-\frac{x^2}{2}} dx 
       \notag\\
    &= \frac{i}{\sqrt{2\pi}} \int_0^\infty \sin(-tx) e^{-\frac{(-x)^2}{2}} dx +
       \frac{i}{\sqrt{2\pi}} \int_{0}^\infty \sin(tx) e^{-\frac{x^2}{2}} dx 
       \notag\\
    &= \frac{-i}{\sqrt{2\pi}} \int_0^\infty \sin(tx) e^{-\frac{x^2}{2}} dx +
       \frac{i}{\sqrt{2\pi}} \int_{0}^\infty \sin(tx) e^{-\frac{x^2}{2}} dx 
       \notag\\
    &= 0\notag
\end{align}

We need to show that $\varphi'(t) = -t \varphi(t)$.

\[\varphi'(t) = \frac{1}{\sqrt{2\pi}} \int -x \sin(tx) e^{-\frac{x^2}{2}} dx \]

Integrating by parts using

\begin{align}
u(x) &= \sin(tx) & \qquad dv(x) &= e^{-\frac{x^2}{2}} (-x) dx \notag\\
du(x) &= t \cos(tx) dx & \qquad v(x) &= e^{-\frac{x^2}{2}} \notag
\end{align}

we have

\begin{align}
\varphi'(t) &= \frac{1}{\sqrt{2\pi}} \sin(tx) e^{-\frac{x^2}{2}} -
    \frac{1}{\sqrt{2\pi}} \int t \cos(tx) e^{-\frac{x^2}{2}} dx \notag\\
            &= \frac{1}{\sqrt{2\pi}} \sin(tx) e^{-\frac{x^2}{2}} -
               t \varphi(t) \notag
\end{align}

Not sure how to remove the first term, except that it is 0 when $t=0$ (TODO).

Assuming we have shown that $\varphi'(t) = -t \varphi(t)$, we can solve
the differential equation below

\begin{align}
\frac{\varphi'(t)}{\varphi(t)} &= -t \notag\\
\ln(\varphi(t)) &= -\frac{t^2}{2}  \qquad\text{(Integrate both sides.)}\notag\\
\varphi(t) &= e^{-\frac{t^2}{2}} \notag%\\
%\varphi(t) e^{\frac{t^2}{2}} &= 1 \notag
\end{align}


%\begin{align}
%d(\varphi(t) e^{\frac{t^2}{2}}) &= \varphi'(t) e^{\frac{t^2}{2}} +
%  \varphi(t) e^{\frac{t^2}{2}} \left(\frac{2}{2} t\right) \notag\\
%    &= -t \varphi(t) e^{\frac{t^2}{2}} + t \varphi(t) e^{\frac{t^2}{2}} 
%      \notag\\
%    &= 0\notag
%\end{align}

%I.e. $d(\varphi(t) e^{\frac{t^2}{2}}) = 0$ for all $t$, which means it is
%constant. Using $\varphi(0) = E[1] = 1$, we have 
%$\varphi(t) e^{\frac{t^2}{2}} = \varphi(0) = 1$ and 
%$\varphi(t) = e^{-\frac{t^2}{2}}$.

\end{enumerate}
\end{section}
\end{document}
